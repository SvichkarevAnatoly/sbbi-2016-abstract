% 12 шрифт, А4 формат
\documentclass[12pt, a4paper]{article} % Класс печатного документа

\usepackage[utf8]{inputenc} % Кодировка исходного текста - utf8
\usepackage[english]{babel} % Язык документа
% отступы (margins)
\usepackage[left=3.5cm, right=2.5cm, top=2.5cm, bottom=2.5cm]{geometry}
\usepackage{mathptmx} % шрифт Times New Roman (для текста и для формул) 

% отмена отступов перед новым абзацем для всего текста
% в Abstract надо вручную отменять отступы
\setlength\parindent{0pt}

% для сноски в виде звёздочки
\makeatletter
\def\@xfootnote[#1]{%
    \protected@xdef\@thefnmark{#1}%
    \@footnotemark\@footnotetext}
\makeatother

% убрать нумерацию страниц
\pagenumbering{gobble}

\begin{document} % Конец преамбулы, начало текста

% изменение заголовка
\renewcommand{\abstractname}{\large \textbf{\uppercase{
    Deep: optimizer with embedded interpreter\\
}}}
\begin{abstract}
    \normalsize % переход к обычному размеру шрифта
    \bigskip % отделение от заголовка одной пустой линией
    \noindent A.~V.~Svichkarev\footnote[*]{Corresponding author}, K.~N.~Kozlov \\
    \noindent
    System biology and bioinformatics lab, IAMM,
    Peter the Great St.Petersburg Polytechnic University,
St.Petersburg, Russia \\
    \noindent e-mail: tolik0393@bionet.nsc.ru \\
\end{abstract}

The problem of parameter estimation
for data-driven models in systems biology
is challenging due to the diversity
of biomedical applications and
the necessity to treat large sets of
heterogeneous data in specialized formats \cite{mendes1998non}.
These tasks usually require
specialized computer environment
and interpreted language for
mathematical modelling.
However, this approach makes
numerical estimation of parameters
of a mathematical model extremely
computationally expensive.

Here we propose
an enhancement to the
Differential Evolution Entirely Parallel (DEEP) method
developed recently~\cite{Kozlov11}.
Our approach reduces the
communication bottleneck
between DEEP and objective function
that describes the deviation
of model solution
from the data.
An interpreter needed to
calculate model solution
is embedded into
DEEP that uses
master-slave architecture.
A master process
organizes an asynchronous queue
of tasks which are run
in the pool of
already started identical interpreters.

The developed approach was
applied to the biological problem
of macromolecular intracellular
trajectory segmentation.
We demonstrated the efficiency
of improved method
in comparison with the previous one.
We performed numerical experiments
with the number of threads
in the range from 1 to 8.
The new method outperformed
the original one by the
factor of 4
in case of using 4
parallel threads and 4
interpreters.
The developed approach can be applied to
such interpreted languages like
R, Octave, Python etc.

Obtained results allowed us
to conclude that
the developed approach
do not lead to significant increase
in RAM allocation thus making
it possible to use
more complex objective functions.
An improved DEEP method ready
to be used as the global optimizer
in the field of systems biology
and bioinformatics.

\vfill % прижать оставшуюся часть к дну страницы
\bibliographystyle{ieeetr} % распространнённый стиль списка литературы
\bibliography{citations} % отображение списка литературы

\end{document} % Конец документа

