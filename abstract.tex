% 12 шрифт, А4 формат
\documentclass[12pt, a4paper]{article} % Класс печатного документа

\usepackage[utf8]{inputenc} % Кодировка исходного текста - utf8
\usepackage[english,russian]{babel} % Язык документа - русский, чтобы дописать по-русски
% \usepackage[english]{babel} % Язык документа
% отступы (margins)
\usepackage[left=3.5cm, right=2.5cm, top=2.5cm, bottom=2.5cm]{geometry}
\usepackage{mathptmx} % шрифт Times New Roman (для текста и для формул) 

% отмена отступов перед новым абзацем для всего текста
% в Abstract надо вручную отменять отступы
\setlength\parindent{0pt}

% для сноски в виде звёздочки
\makeatletter
\def\@xfootnote[#1]{%
    \protected@xdef\@thefnmark{#1}%
    \@footnotemark\@footnotetext}
\makeatother

% убрать нумерацию страниц
\pagenumbering{gobble}

\begin{document} % Конец преамбулы, начало текста

% изменение заголовка
\renewcommand{\abstractname}{\large \textbf{\uppercase{
    Deep: optimizer with embedded interpreter\\
}}}
\begin{abstract}
    \normalsize % переход к обычному размеру шрифта
    \bigskip % отделение от заголовка одной пустой линией
    \noindent A.~V.~Svichkarev\footnote[*]{Corresponding author}, K.~N.~Kozlov \\
    \noindent
    System biology and bioinformatics lab, IAMM,
    Peter the Great St.Petersburg Polytechnic University,
St.Petersburg, Russia \\
    \noindent e-mail: tolik0393@bionet.nsc.ru \\
\end{abstract}

The problem of parameter estimation
for data-driven models in systems biology
is challenging due to the diversity
of biomedical applications and
the necessity to treat large sets of
heterogeneous data in specialized formats \cite{mendes1998non}.
Поэтому математическое моделирование
часто проводится в специализированной среде,
использующей какой-либо интерпретатор.
Это однако ведёт к росту
вычислительных затрат
при численном определении
параметров модели
путём подгонки к данным.

Здесь мы предлагаем улучшить
взаимодействие между
Differential Evolution Entirely Parallel (DEEP)
method develop recently~\cite{Kozlov11}
and целевой функцией сравнения
решений модели с данными.
Было достигнуто увеличение
производительности DEEP
путём встраивания интерпретатора.
Решение основано на применении
архитектуры master-slave,
в которой master процесс
организует ассинхронную очередь задач,
выполняющихся в пуле готовых интерпретаторов.

Полученная реализация использует
функции и структуры библиотеки GLib
и была протестирована
на прикладной биологической задаче
сегментации движения макромолекул в клетках.
Показана эффективность улучшения интеграции
по сравнению со преждней реализации,
следствием чего является
увеличение производительности DEEP.

Реализованный подход применим для использования
таких интерпретируемых языков,
как R, Octave, Python and etc.

В заключение стоит сказать,
не просели по памяти.
можно использовать более сложные функции качества.
Улучшение DEEP ведёт к увеличению
количества проверяемых гипотез
в области системной биологии
и биоинформатике.

\vfill % прижать оставшуюся часть к дну страницы
\bibliographystyle{ieeetr} % распространнённый стиль списка литературы
\bibliography{citations} % отображение списка литературы

\end{document} % Конец документа

