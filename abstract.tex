% 12 шрифт, А4 формат
\documentclass[12pt, a4paper]{article} % Класс печатного документа

\usepackage[utf8]{inputenc} % Кодировка исходного текста - utf8
\usepackage[english,russian]{babel} % Язык документа - русский, чтобы дописать по-русски
% \usepackage[english]{babel} % Язык документа
% отступы (margins)
\usepackage[left=3.5cm, right=2.5cm, top=2.5cm, bottom=2.5cm]{geometry}
\usepackage{mathptmx} % шрифт Times New Roman (для текста и для формул) 

% отмена отступов перед новым абзацем для всего текста
% в Abstract надо вручную отменять отступы
\setlength\parindent{0pt}

% для сноски в виде звёздочки
\makeatletter
\def\@xfootnote[#1]{%
    \protected@xdef\@thefnmark{#1}%
    \@footnotemark\@footnotetext}
\makeatother

% убрать нумерацию страниц
\pagenumbering{gobble}

\begin{document} % Конец преамбулы, начало текста

% изменение заголовка
\renewcommand{\abstractname}{\large
    Method development of Differential Evolution
    for search mathematical model's options}
\begin{abstract}
    \normalsize % переход к обычному размеру шрифта
    \bigskip % отделение от заголовка одной пустой линией
    \noindent A.~V.~Svichkarev\footnote[*]{Corresponding author}, K.~N.~Kozlov \\
    \noindent
    System biology and bioinformatics lab, IAMM,
    Peter the Great St.Petersburg Polytechnic University,
St.Petersburg, Russia \\
    \noindent e-mail: tolik0393@bionet.nsc.ru \\
\end{abstract}

An important class of programs
for mathematical modeling are
software packages for solution
of the inverse problem of mathematical modeling.
Main aim is formulated as
minimization the quality functional
with limitations of the parameters.
There are a lot of applications
in system biology and medicine.

Differential Evolution Entirely Parallel (DEEP)
is modification of stochastic method optimization
for function minimization \cite{Kozlov11}.
Original Differential Evolution (DE)
is a stochastic parallel direct search
evolution strategy optimization method
which was proposed by Storn and Price in 1995 \cite{Storn95}. 

Целью данной работы являлось улучшение взаимодействия
между методом оптимизации
и функционалом качества
для увеличения производительности
работы программы DEEP.

Математические модели в биоинформатике
в большинстве случаев создаются
на интерпретируемых языках программирования,
к примеру на языке R для статистической обработки данных.
Для поиска параметров таких моделей
требуется многократно запускать
и инициализировать интерпретатор
для вычисления функционала.
Это требует больших вычислительных затрат.
Однако интерпретатор может быть встроен
в программу DEEP для
многократного повторного использования.

Было спроектировано и реализовано
встраивание интерпретатора языка R
в DEEP с использованием
функций и структур библиотеки GLib.
Решение основано на применении
ассинхронной очереди задач и пула потоков.
Полученная реализация была
протестирована на прикладной биологической задаче
сегментации движения макромолекул в клетках.
Показана эффективность улучшения интеграции
по сравнению со преждней реализации,
следствием чего является
увеличение производительности DEEP.
Развитие DEEP ведёт к увеличению
количества проверяемых гипотез.

\vfill % прижать оставшуюся часть к дну страницы
\bibliographystyle{ieeetr} % распространнённый стиль списка литературы
\bibliography{citations} % отображение списка литературы

\end{document} % Конец документа

