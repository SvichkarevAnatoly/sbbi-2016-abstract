% 12 шрифт, А4 формат
\documentclass[12pt, a4paper]{article} % Класс печатного документа

\usepackage[utf8]{inputenc} % Кодировка исходного текста - utf8
\usepackage[english]{babel} % Язык документа
% отступы (margins)
\usepackage[left=3.5cm, right=2.5cm, top=2.5cm, bottom=2.5cm]{geometry}
\usepackage{mathptmx} % шрифт Times New Roman (для текста и для формул) 

% отмена отступов перед новым абзацем для всего текста
% в Abstract надо вручную отменять отступы
\setlength\parindent{0pt}

% для сноски в виде звёздочки
\makeatletter
\def\@xfootnote[#1]{%
    \protected@xdef\@thefnmark{#1}%
    \@footnotemark\@footnotetext}
\makeatother

% убрать нумерацию страниц
\pagenumbering{gobble}

\begin{document} % Конец преамбулы, начало текста

% изменение заголовка
\renewcommand{\abstractname}{\large \textbf{\uppercase{
    Title}}}
\begin{abstract}
    \normalsize % переход к обычному размеру шрифта
    \bigskip % отделение от заголовка одной пустой линией
    \noindent I.~A.~Ivanov\footnote[*]{Corresponding author}, N.~A.~Sidorov \\
    \noindent Institute of Cytology and Genetics SB RAS, Novosibirsk, Russia\\
    \noindent e-mail: ivanov@bionet.nsc.ru \\
\end{abstract}

Place text here \cite{ref1}.

\vfill % прижать оставшуюся часть к дну страницы
\bibliographystyle{ieeetr} % распространнённый стиль списка литературы
\bibliography{citations} % отображение списка литературы

\end{document} % Конец документа

